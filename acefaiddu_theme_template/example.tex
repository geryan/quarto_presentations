%\documentclass{beamer}
\documentclass[12pt,aspectratio=169,presentation]{beamer}

\usepackage{amsmath}
\usepackage{unicode-math}
\usepackage{fontspec}

%% NOTE: would be nice to use a font that supports `Numbers=SlashedZero`.
\defaultfontfeatures{Scale=MatchLowercase,Ligatures=TeX}
\setsansfont[Numbers={Lining}]{Open Sans}
\setmainfont[Numbers={Lining}]{Open Sans}
\setmathfont{Latin Modern Math}

\title{There Is No Largest Prime Number}
\date[ISPN ’80]{27th International Symposium of Prime Numbers}
\author[Euclid]{Euclid of Alexandria\\\texttt{euclid@alexandria.edu}}

\usetheme{acefa-iddu}

\begin{document}

\acefaiddutitleframe

\begin{frame}{There Is No Largest Prime Number}
  \begin{theorem}
    There is no largest prime number. \end{theorem}
  \begin{enumerate}
  \item<1-| alert@1> Suppose \(p\) were the largest prime number.
  \item<2-> Let \(q\) be the product of the first \(p\) numbers.
  \item<3-> Then \(q+1\) is not divisible by any of them.
  \item<1-> But \(q + 1\) is greater than \(1\), thus divisible by some prime
    number not in the first \(p\) numbers.
  \end{enumerate}
\end{frame}

\begin{frame}{A longer title}
  \begin{itemize}
  \item one
  \item two
  \end{itemize}
\end{frame}

\begin{frame}{Some quote}{Test sub title}
  \begin{block}{Example block}
    {\large ``You miss 100\% of the shots you don't take.'' – Wayne Gretzky}
    \vspace{\baselineskip}

    \hspace*\fill{\small— Michael Scott}
  \end{block}
\end{frame}

\end{document}
